% !TEX root = robobrain.tex

% \section{Technical Challenges}
\section{Overview}
\label{overviewPaper}

\robobrain{} is a never ending learning system  that  continuously incorporates
new knowledge from  its partner projects and from different Internet sources.
One of the functions of \robobrain{} is to represent the knowledge from various sources as a graph,
as shown in  Figure~\ref{fig:graph}. The nodes of the graph represent concepts and edges represent the
relations between them. The connectivity of the graph is increased through a set of graph operations
that allow additions, deletions and updates to the graph. As of the date of this submission,
\robobrain{} has successfully
connected knowledge from sources like WordNet, ImageNet, Freebase, OpenCyc,
 parts of Wikipedia and other partner projects. These knowledge sources provide lexical knowledge, grounding of concepts into images and common sense facts about the world.

The knowledge from the partner projects and Internet sources can sometimes be erroneous. \robobrain{}
handles inaccuracies in  knowledge by maintaining beliefs over the correctness of the concepts and
relations. These beliefs depend on how much \robobrain{} trusts a given source of knowledge, and also the
feedback it receives from crowd-sourcing (described below). For every incoming knowledge, \robobrain{}
also makes a sequence of decisions on whether to form new nodes, or edges, or both. Since the
knowledge carries semantic meaning \robobrain{} makes many of these decisions based on the
contextual information that it gathers from nearby nodes and edges. For example, \robobrain{} resolves
polysemy using the context associated with nodes. Resolving polysemy is important because a `plant'
could mean a `tree' or an `industrial plant' and merging the nodes together will create errors in the
graph.


\robobrain{} incorporates supervisory signals from humans in the form of crowd-sourcing feedback. This
feedback allows \robobrain{} to update its beliefs over the correctness of the knowledge, and to modify the
graph structure if required. While crowd-sourcing feedback was used in some previous works as
means for data collection (e.g.,~\citep{imagenet2009,Russell08}), in \robobrain{} they serve as supervisory
signals that improve the knowledge engine. \robobrain{} allows user interactions at multiple levels: (i)~
Coarse feedback: these are binary feedback where a user can ``Approve'' or ``Disapprove'' a concept
in \robobrain{} through its online web interface; (ii) Graph feedback: these feedback are elicited on 
\robobrain{} \textit{graph visualizer}, where a user modifies the graph by adding/deleting nodes or edges; (iii)
Robot feedback: these are the physical feedback given by users directly on the robot.

In this paper we discuss different aspects of \robobrain{}, and show how \robobrain{} serves as a knowledge layer for the robots. In order to support knowledge sharing, learning, and crowd-sourcing feedback we develop a large-scale distributed system. We describe the architecture of our  system in Section~\ref{sec:system}. In Section~\ref{sec:raquel} we describe the robot query library, which allow robots to interact with \robobrain{}.  Through experiments we show that robots can use \robobrain{} \textit{as-a-service} and that knowledge sharing through \robobrain{} improves existing robotic applications. We now present a formal definition of our Robot Knowledge Engine and the graph.
