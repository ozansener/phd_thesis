% !TEX root = robobrain.tex

\section{Crowdsourcing}

One key component in RoboBrain is the use of crowd-sourcing.   While 
previous works have focussed on using crowd-sourcing for data collection and
labeling (e.g., \cite{blah,blah2}), our work focusses on taking crowd-sourced
feedback at many levels:
\begin{itemize}

\item \emph{Weak feedback.}  We have found that it is easiest for users
to provide a binary ``Approve''/``Dispprove'' feedback while they are browsing
online. Our RoboBrain system encourages users to provide such feedback
which is used for  several purposes such as estimating the belief on the nodes
and edges in RoboBrain.

% \todo{Talk about feedback on feeds.}

\item \emph{Feedback on graph.}  We present the graph to the users on the Internet
and smartphones, where they can give feedback on the correctness of a node
or an edge in the graph.  Such feedback is stored for inferring the latent graph.
The key novelty here is that the system shows the proposed ``learned" concepts to the 
users --- thus getting feedback not at the data level but at the knowledge level.

% \todo{Feedback directly on graph.}

\item  \emph{Feedback from partner projects.}   There are several partner 
projects to RoboBrain, such as Tell Me Dave, PlanIt, and so on.  These projects
have a crowd-sourced system where they take feedback, demonstrations,
or interaction data from the users. Such data as well as the learned knowledge 
could be shared with RoboBrain.

% \todo{Feedback via partner projects individual crowd-sourcing. (RoboBrain is a meta-effort.)}

\end{itemize}
