% !TEX root = rCRF.tex


\begin{abstract}
For assistive robots, anticipating the future actions of humans is an essential task.
% Successful anticipation of the future activities of humans,
This requires modelling both the evolution of the activities over time and the rich relationships between humans and the objects. Since the future activities of humans are quite ambiguous, robots need to assess all the future possibilities in order to choose an appropriate action. Therefore, a successful anticipation algorithm  needs to compute all plausible future activities and their corresponding probabilities.

In this paper, we address the problem of efficiently computing beliefs over future human activities from RGB-D videos. We present a new recursive algorithm that we call Recursive Conditional Random Field (rCRF) which can compute an accurate belief over a temporal CRF model. We use the rich modelling power of CRFs and describe a computationally tractable inference algorithm based on Bayesian filtering and structured diversity. In our experiments, we show that incorporating belief, computed via our approach, significantly outperforms the state-of-the-art methods, in terms of accuracy and computation time.
\end{abstract}

%While Bayesian filtering over a generative model is good for integrating the measurements over time for computing future state beliefs, they do not directly apply when the state and the measurements have a rich structure. In such a case, a Condition Random Field (CRF) can model the structure well. However, its discriminative nature prevents the computation of the beliefs.
%modelling activities in a 3D environment
%In order to model activities in a 3D environment, we not only need to
%based approaches are


% relationships between activities, humans and o

% We consider the problem of estimating a belief over a variable of interest from a sequential visual data using a rich CRF model. We focus our application and experiments on the problem of activity analysis from RGB-D data due to its importance for robotics and surveillance areas. In order to compute the belief, we present a new recursive approach which we call Recursive Conditional Random Field (rCRF). rCRF combines the rich model of the CRF with the recursive nature of the Bayesian filtering in order to represent an accurate belief. We make two carefully tailored approximation via structured diversity and the Jensen inequality in order to enable tractable computation. In our experiments, we show that incorporating a belief computed via our approach significantly outperforms the state-of-the-art methods in the tasks of human activity detection and anticipation. The key contribution of our work is to show that even with large output spaces (such as a CRF), it is possible to estimate the full belief accurately with a recursive approach based on Bayesian filtering.
